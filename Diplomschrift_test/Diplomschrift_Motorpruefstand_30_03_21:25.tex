%%%%%%%%%%%%%%%%%%%%%%%%%%%%%%%%%%%%%%%%%%%%%%%%%%%%%%%%%%%%%%%%%%%%%%%%%%%%%
%	XH Herstellungsprozess-Shellscript @15Apr17:
%	--------------------------------------------
%   a)  wenn keine 'ps'-Macros genutzt werden:
%	#!/bin/sh
%	pdflatex $ifn
%	pdflatex $ifn
%   b)  mit 'ps' Macros (zB. 'pspicture') im Source:
%	#!/bin/sh
%	ifn="MeinLatexFile.tex"
%	latex $ifn
%	fn2="$(echo $ifn|sed s/.tex/.dvi/)"
%	fn3="$(echo $ifn|sed s/.tex/-pics.pdf/)"
%	  ### now dvipdf  $fn2 into $fn3 Container ...
%	dvipdf $fn2  $fn3
%	#dvipdf $(echo $ifn|sed s/.tex/.dvi/)
%	  ### now pdflateXing $ifn ... (zweimal - fuers Inhaltsverzeichnis)
%	pdflatex $ifn
%	pdflatex $ifn
%%%%%%%%%%%%%%%%%%%%%%%%%%%%%%%%%%%%%%%%%%%%%%%%%%%%%%%%%%%%%%%%%%%%%%%%%%%%%
\listfiles			%lists included files while processing 'pdflatex'
\documentclass[12pt,a4paper]{article}
  %\usepackage{etex}		%gegen 'no more room for new dimen...' error bei 'tikz' xh@RaE1

	% encoding:
  %%\usepackage[latin1]{ucs}
  %%\usepackage[latin1]{inputenc}
  %%\usepackage[ansinew]{inputenc}
  %%\usepackage[cp850]{inputenc}
  %\usepackage[utf8x]{inputenc}
\usepackage[utf8]{inputenc}
\usepackage[ngerman]{babel}
\usepackage[T1]{fontenc}

%\usepackage{geometry}						geht nit
%\geometry{bottom=-20mm}

  %\usepackage{amssymb}
\usepackage{amsmath}
  %\usepackage{extarrows}	%\xleftrightarrow[obentext]{untentext}
\usepackage{wasysym}
\usepackage{pxfonts}
\usepackage{verbatim}
\usepackage{alltt}
\usepackage{moreverb}
\usepackage{graphicx}
\usepackage{wrapfig}
\usepackage{subfigure}
  %\usepackage{theorem}
  %\usepackage[dvips]{color}
  %\usepackage{lmodern}
  %\usepackage{textcomp}
\usepackage{multicol}		% 2-, 3-, ... -spaltige Formatierung mit 'multicols'
\usepackage{multirow}		% fuer 'tabular' - Tabellen
 %\usepackage{makeidx}
\usepackage{imakeidx}		% erspart dieses externe Stichwortprogramm
 \usepackage{pdfpages}	% fuer 'includepdf' (stattdessen nimmi meistens 'includegraphics[page=1,...]')%=======================????????????????????????????????????
\usepackage{mdwlist}		% f. 'compact lists' "itemize*", "enumerate*", "description*"
  %\usepackage{ulem}		%... produziertma nFehler ban 'latex' run
\usepackage{longtable}		% fuer tabellen ueber mehrere Seiten
\usepackage{xcolor}
\definecolor{lltgy}{gray}{0.96}	% selbstdefinierte Farben
\definecolor{ltgy}{gray}{0.91}
\definecolor{dkgn}{rgb}{0.0,0.7,0.0}
\definecolor{dkbu}{rgb}{0.0,0.0,0.6}
\definecolor{ddkbu}{rgb}{0.0,0.0,0.45}
%\definecolor{htllogo}{cmyk}{95,43,0,39}
\definecolor{htllogo}{RGB}{7,88,155}

\usepackage{ifthen}		% fuer 'ifthenelse{}{}{}'
\usepackage{ifpdf}		% fuer {\ifpdf ... \fi}
\usepackage{listings}
\lstset{language=C}
\lstset{basicstyle=\footnotesize}
  %\lstset{basicstyle=\small}
  %\lstset{basicstyle=\normalsize}
\lstset{backgroundcolor=\color{ltgy}}%=======================????????????????????????????????????
\lstset{showstringspaces=false}
\lstset{breaklines=true}
  %\lstset{tabsize=4}
\lstset{morecomment=[l][\color{dkgn}]{\%},%
	morecomment=[s][\color{dkgn}]{/*}{*/}}
\lstset{numbers=left}

\usepackage{fancyhdr}
  %\usepackage{framed}		%'\begin{framed}' ... '\end{framed}', schautAusWiePartezettel:-)
\usepackage{hyphenat}		%fuer '\hyph{}'
  %\usepackage{lastpage}	%fuer '\pageref{LastPage}' - **funzt nid bei allen**
\usepackage{url}		%fuer '\url{...}'

% lscape oder pdflscape: ('landscape' == Querformat)
\usepackage{lscape}
  %\usepackage{pdflscape}
\usepackage{rotating}		%f. 'rotate' und 'turn'
\usepackage[active]{pst-pdf}
\usepackage{pst-circ}
\usepackage{pst-plot}
\usepackage{pst-uml}
  %\usepackage{calc}
\usepackage{fp}
  %\usepackage[official]{eurosym}
\usepackage[gen]{eurosym}

	% YHs Raender links 30mm rechts 25mm einstellen:
\setlength{\hoffset}	{30mm-1in}
\setlength{\oddsidemargin}{0pt}		%bei doppelseitigem Druck umstellen!
\setlength{\textwidth}	{\paperwidth-55mm}

\setlength{\topmargin}	{2pt} %{0pt}
\addtolength{\voffset}  {-23mm} %{-16.2mm}
\addtolength{\textheight}{39mm}

\setcounter{tocdepth}{4}		%bringt auch 'paragraph{titel}' ins Inhaltsverzeichnis
%
\definecolor{ydkbu}{rgb}{0.0,0.0,0.6}	% YHs blaue Schriftfarb
\newcommand{\yhbu}[0]{\color{ydkbu}}	% Macro fuer schreibfaulen XH
%\newcommand{\yhbu}[0]{\color{ydkbu}\usefont{T1}{laess}{m}{n}}	% Macro fuer schreibfaulen XH
\definecolor{corrclr}{rgb}{0.7,0.2,0.2}		% XHs Korrekturen-Farb ...
\newcommand{\korr}[0]{\color{corrclr}\fontsize{8pt}{9pt}\selectfont\bf} %plus Faulheitsmacro
\makeindex

	%/* Line Spacing: */
\usepackage{setspace}


	%/* selbst-definierte Macros: */
\newcommand{\cmnt}[1]{}			%eigene Kommentier-Funktion \cmnt{ ...Kommentar... }
\newcommand\tbs{\textbackslash}		%'\textbackslash{}' isma z'long zan tippen ;-)
\newcommand{\xilist}[6]{		%XHs eigenes Auflistungs-Format
	\noindent\\[#3mm]\protect\begin{list}{#1}{\setlength\topsep{-4mm}%
	\setlength\partopsep{-1.0mm}\setlength\itemsep{#4mm}\setlength\leftmargin{#2mm}%
	\setlength\itemindent{-0.0mm}\setlength{\baselineskip}{0.9\baselineskip} }%
	#6%
	\protect\end{list}\hfill\\[#5mm]%
}

	%/* eineinhalbzeilige Formatierung: */
% \newcommand{\mylinespacing}[0]{\singlespace}	% 1,0-ZeilenAbstand
\newcommand{\mylinespacing}[0]{\onehalfspace}	% 1,5-ZeilenAbstand
% \newcommand{\mylinespacing}[0]{\doublespace}	% 2,0-ZeilenAbstand


	%/* serifenlose Schrift-Grundeinstellung: */
% /*Font Family:*/
%\renewcommand*{\familydefault}{\rmdefault}	%klassisches 'Roman' (statt MicroMurx...)
\renewcommand*{\familydefault}{\sfdefault}	%klassisches 'Helvetica' statt 'Murx-Arial'


%====================================================================================
%\clearpage\vfill\newpage{}
%====================================================================================

%
%===================================================
%
\begin{document}
%\addtocontents{toc}{\protect\begin{multicols}{2}} %-fuer mehrspaltiges Inh.Verz
\renewcommand{\listfigurename}{}
\renewcommand{\listtablename}{}

\newcommand\logoB[1]{%
	%dieses Macro 'logoB' zeichnet das "neue" HTL Logo mithilfe der
	% 'ps-tricks' Pakete/Anweisungen; Parameter#1 bestimmt die "Dicke"
	% der Balken; die "Groesse" bitte mit '\scalebox{factor}{logoB{0.12}}',
	% die Grundlinie mit '\raisebox{pos}{logoB{0.12}}' einstellen;
	% die Farbgebung spezifiziert man HIER:
  \definecolor{lobu}{rgb}{0.05,0.05,0.50}
  \definecolor{hibu}{rgb}{0.20,0.20,0.70}
  \definecolor{loye}{rgb}{0.85,0.75,0.36}
  \definecolor{hiye}{rgb}{0.99,0.92,0.00}
  \definecolor{logn}{rgb}{0.00,0.65,0.20}
  \definecolor{hign}{rgb}{0.00,0.79,0.30}
  \definecolor{lord}{rgb}{0.66,0.00,0.00}
  \definecolor{hird}{rgb}{0.89,0.00,0.00}
  %
  \resizebox{11.5mm}{!}{%
  \begin{pspicture}[showgrid=false](-1,-1)(1,1)
	\SpecialCoor	%das erlaubt PS -Berechnungen mit dem '!'; hier zur "DickenSkalierung"
	\pspolygon[linewidth=0.1pt,linestyle=none,fillcolor=lobu,fillstyle=solid]%
		(-#1, -1.00)( #1, -1.00)( 1.00, -#1)(! 1.00 #1 2 mul sub -#1)
	\pspolygon[linewidth=0.1pt,linestyle=none,fillcolor=hibu,fillstyle=solid]%
		(! 1.00 #1 2 mul sub          -#1)(! 1.00 #1 3 mul sub   0.00)%
		(! -#1         -1.00 #1 2 mul add)(-#1,-1.00)

	\pspolygon[linewidth=0.1pt,linestyle=none,fillcolor=hiye,fillstyle=solid]%
		( 1.00, -#1)( 1.00, #1)( #1, 1.00)(! #1   1.00 #1 2 mul sub)
	\pspolygon[linewidth=0.1pt,linestyle=none,fillcolor=loye,fillstyle=solid]%
		(! #1    1.00 #1 2 mul sub)(! 0.00   1.00 #1 3 mul sub)%
		(! 1.00 #1 2 mul sub   -#1)( 1.00, -#1)

	\pspolygon[linewidth=0.0pt,linestyle=none,fillcolor=hign,fillstyle=solid]%
		( #1, 1.00)( -#1, 1.00)(-1.00, #1)(! -1.00 #1 2 mul add   #1)
	\pspolygon[linewidth=0.0pt,linestyle=none,fillcolor=logn,fillstyle=solid]%
		(! -1.00 #1 2 mul add   #1)(! -1.00 #1 3 mul add    0.00)%
		(! #1    1.00 #1 2 mul sub)( #1, 1.00)

	\pspolygon[linewidth=0.1pt,linestyle=none,fillcolor=lord,fillstyle=solid]%
		(-1.00, #1)(-1.00, -#1)(-#1, -1.00)(! -#1    -1.00 #1 2 mul add)
	\pspolygon[linewidth=0.1pt,linestyle=none,fillcolor=hird,fillstyle=solid]%
		(! -#1   -1.00 #1 2 mul add)(! 0.00   -1.00 #1 3 mul add)%
		(! -1.00 #1 2 mul add    #1)(-1.00, #1)
	\NormalCoor
  \end{pspicture}%
  }%
}

\newcommand{\HtlHeader}[0]{%
	%\hspace*{-11mm}%
	%\raisebox{-1mm}{\logoB{0.12}}%
	{\ifpdf
	 \includegraphics[width=10.3mm]{pics/logoBpdf.pdf}
	\fi}
	\hspace*{2mm}%
	\parbox[b]{110mm}{\flushleft
		{\fontsize{20pt}{20pt}\selectfont\bf HTL}
		{\fontsize{16.2pt}{16.2pt}\selectfont\color{htllogo}\bf anichstra"se}
		\\[-4.05mm]{\color{darkgray}\rule{110mm}{0.5pt}}
		\\[-2.24mm]{\fontsize{7pt}{7pt}\selectfont\color{darkgray}
			Elektronik $\cdot$ Elektrotechnik $\cdot$
			Maschinenbau $\cdot$ Wirtschaftsingenieure
			\rule{0pt}{0mm}
		%\vspace*{1.1mm}
		}
	}%
	\hspace*{5mm}%
	\raisebox{-0.2mm}{\ifpdf \includegraphics[width=25mm]{pics/HTLgenlogo02}\fi}
	\\[-1.5mm]\rule{\textwidth}{0.5pt}
	%\hfill
}%HtlHeader


%/*Header-Einstellung*/
\pagestyle{fancy}
\fancyhf{}
\renewcommand{\sectionmark}[1]{\markright{#1}}
\renewcommand{\subsectionmark}[1]{\markright{#1}}
\renewcommand{\subsubsectionmark}[1]{\markright{#1}}
\lhead{\HtlHeader}
\chead{}
\rhead{}
\lfoot{Simon Jehle, Patrick Krismer}
\cfoot{\thesection-\rightmark}
%\cfoot{\thesubsubsection-\rightmark}
\rfoot[\thepage]{\thepage/\pageref{LastPage}}
\setlength{\headwidth}	{1.0\textwidth}
\setlength{\headheight}{14mm}					%14mm sonsst sein die Bilder im Header weg
\renewcommand{\headrulewidth}{0.0pt}
\renewcommand{\footrulewidth}{0.33pt}


\addtocounter{page}{0}
%
%====================================================================================
%
 \begin{center}
   \begin{minipage}{\linewidth}
   \begin{center}
	\vspace*{-14mm}
	%\HtlHeader{}
	\noindent%
	\\[35mm]{\fontsize{25pt}{25pt}\selectfont\bf DIPLOMARBEIT}
	\\[19mm]{\fontsize{20pt}{20pt}\selectfont\textbf{\textsc{Flugmodellmotor-Prüfstand}}}
	\\[15mm]{\fontsize{12.4pt}{12.4pt}\selectfont\bf
		Höhere Technische Bundeslehr- und Versuchsanstalt Anichstra"se}
	\\[ 5mm]\rule{132mm}{1.0pt}
	\\[ 4mm]{\fontsize{12.4pt}{12.4pt}\selectfont\bf Abteilung}
	%\\[ 5mm]{\fontsize{12.4pt}{12.4pt}\selectfont\bf Elektronik \& Technische Informatik}
	\\[ 5mm]{\fontsize{16pt}{16pt}\selectfont
		\textbf{\textsc{Elektronik und technische Informatik}}}
	\\[24mm]{\hspace*{2mm}\parbox{154mm}{\fontsize{12.4pt}{12.4pt}\selectfont
	  \parbox[t]{75mm}{
		Ausgef"uhrt im Schuljahr 2019/20 von:
		\\[5.0mm]Simon Jehle 5BHel
		\\[2.5mm]Patrick Krismer 5BHel
	  }
	  \hspace*{6mm}
	  \parbox[t]{55mm}{
		Betreuer/Betreuerin:
		\\[5.0mm]Dipl.-Ing. Christoph Schönherr
		\\[5.0mm]Dipl.-Ing. Christian Fischer
		
	  }
%	  \\[12mm]{Projektpartner: {\color{blue}Dipl.-Ing. Christian Fischer, HTL-Anichstraße}, Innsbruck}
	  \\[14mm]{Innsbruck, am 01.04.2020}
	  \\[16mm]\rule{150mm}{0.5pt}
	  \\[ 8mm]
	  \parbox[t]{75mm}{
		Abgabevermerk:
		\\[3.25mm]Datum:
	  }
	  \hspace*{6mm}
	  \parbox[t]{50mm}{
		Betreuer/in:
	  }
	}}
   \end{center}\hfill
   \end{minipage}
 \end{center}
%
%====================================================================================
%
\newpage
\subsection*{Danksagung}
	An dieser Stelle möchten wir uns bei allen Personen bedanken, die uns während der gesamten Diplomarbeit 
	unterstützt haben und zu deren Gelingen beigetragen haben. \\\\
%	XH sagt er will nit bedankrt werden	
\noindent
	Bedanken möchten wir uns bei den {\bf Fachlehrern} und {\bf Professoren}, die uns durchgehend Ratschläge gaben ...\\\\
\noindent
	Ebenfalls möchten wir uns bei den {\bf Mitarbeitern der Materialbeschaffung} bedanken. Durch die hervorragende 
	Organisation in den Magazinen konnten alle nötigen Bauteile schnell erhalten werden. \\\\
\noindent
	Abschließ end möchten wir uns auch recht herzlich bei {\bf unseren Eltern} bedanken, die uns dei gesamte Zeit 
	emotionalen Rückhalt gaben ... 
%
%====================================================================================
%
\newpage
\subsection*{Gendererklärung}
	Aus Gr"unden der besseren Lesbarkeit wird in dieser Diplomarbeit die Sprachform
	des generischen Maskulinums angewendet. Es wird an dieser Stelle darauf
	hingewiesen, dass die ausschlie"sliche Verwendung der m"annlichen Form
	geschlechtsunabh"angig verstanden werden soll.
%
%====================================================================================
%
\newpage
%\subsection*{Kurzfassung / Abstract}
%\subsubsection*{Deutsch}
%	Die vorliegende Diplomarbeit beschreibt die weitere Verbesserung eines Elektromotor-Prüfstandes. \\
%	Ein bereits, aus vorherigen Diplomarbeiten unter der Aufsicht von Dipl. Ing Christian Fischer, 
%	vorhandener Prüfstand für Klein-Elektromotoren wird dahingehend verändert, dass technische Mängel 
%	ausgebessert werden und der gesamte Messvorgang automatisiert abläuft. \\\\
%	Darunter fällt die Regelung der Drehzahl des Elektromotors und die Einstellung der mechanischen Belastung 
%	durch eine Wirbelstrombremse. \\
%
%	Gemessen werden die Eingangs- und Ausgangsleistungen, die im Zusammenhang mit dem gemessenen Drehmoment
%	den {\bf Wirkungsgrad} des Motors ergeben. \\
%	Dieser wird wiederum im Zusammenhang mit der {\bf Drehzahl} und der {\bf mechanischen Belastung} angezeigt, um den 
%	optimalen Wirkungsgrad des Motors ermitteln zu können. \\\\
%
%	Die Messungen werden durch divere Bauteile wie Hallsensoren, einer Waage, einer Lichtschranke u.w. durchgeführt. 
%	Die Rohdaten werden in dem 'Embedded System' {\bf Aruino MEGA 2560} verarbeitet. 
%	Dieser sendet dann die Messwerte zu einem {\bf Raspberry Pi 3B}, welcher die Werte speichert und auf der Bedienwebseite 
%	in Tabellenform anzeigt. \\\\
%
%	Der Prüfstand ist insbesondere für den DC-Elektromotor: {\bf Dualsky XM6360EA-10	Outrunner Brushless Motor} konzipiert. 
%	Dieser wird als Antrieb für den Propeller eines RC-Modellflugzeuges verwendet. 
%
\subsection*{Kurzfassung / Abstract}
\subsubsection*{Deutsch}
  Die vorliegende Diplomarbeit beschreibt die weitere Verbesserung eines Elektromotor-Prüfstandes. \\
  Ein bereits aus vorherigen Diplomarbeiten, unter der Aufsicht von Dipl. Ing. Christian Fischer,
  vorhandener Prüfstand für Klein-Elektromotoren wird dahingehend verändert, dass technische Mängel 
  ausgebessert werden und der gesamte Messvorgang automatisiert abläuft. \\\\
  Darunter fällt die Regelung der Drehzahl des Elektromotors und die Einstellung der mechanischen Belastung 
  durch eine Wirbelstrombremse. \\
%
  Gemessen wird die das Drehmoment und die Drehzahl des Motors, sodass die Ausgangsleistung berechnet werden kann, 
  sowie die Eingangsleistung, sodass der {\bf Wirkungsgrad} des Motors berechnet werden kann. \\
  Dieser wird wiederum im Zusammenhang mit der {\bf Drehzahl} und der {\bf mechanischen Belastung} angezeigt, um den 
  Wirkungsgrad des Motors bei verschiedenen Bedingungen ermitteln zu können. \\\\
%
  Die Messungen werden durch divere Bauteile wie Hallsensoren, einer Waage, einer Lichtschranke und weiteren durchgeführt. 
  Die Rohdaten werden in dem 'Embedded System' {\bf Arduino MEGA 2560} verarbeitet. 
  Dieser sendet dann die Messwerte zu einem {\bf Raspberry Pi}, welcher sie auf der Bedienwebseite 
  in Tabellenform anzeigt und, falls ein Messdurchgang durchgeführt wird, die Werte speichert. \\\\
%
  Der Prüfstand ist insbesondere für den DC-Elektromotor: {\bf Dualsky XM6360EA-10  Outrunner Brushless Motor} konzipiert. 
  Dieser wird als Antrieb für den Propeller eines RC-Modellflugzeuges verwendet. 
%
%====================================================================================
%
\newpage
\subsubsection*{Englisch}
  This thesis describes the further improvement of an Electromotor-testing-bench.\\
  An already existing testing bench, from previous theses under the supervision by Dipl. Ing. Christian Fischer, 
  is being changed, for an improvement of technical faults and for the automatisation of the entire measuring-process.\\\\
  This includes the controlling of the rotational speed of the motor and the adjustment of the mechanical load, using an
  eddy current brake.
%
  Measurements are being taken of the torque and the rotational speed of the motor, which are being used for calculating
  the output-power, and the input-power, with which the {\bf efficiency factor} of the motor can be calculated. \\
  This factor is, like the {\bf rotational speed} and the {\bf mechanical load}, displayed, to work out the efficiency factor
  of the motor at different conditions. \\\\
%
  The measurements are being carried out by various components, like Hall-sensors, a scale, a photoelectric barrier and others.
  The raw measurements are being processed in the 'Embedded System' {\bf Arduino MEGA 2560}.
  It sends the processed measurements to a {\bf Raspberry Pi}, which displays them on the Interface-Website in table-form 
  and, if a measuring-process is being carried out, saves the measured values. \\\\
%
  The testing bench was developed with the {\bf Dualsky XM6360EA-10 Outrunner Brushless Motor} in mind.
  This motor is being used as the drive for an RC model airplane.
%
%====================================================================================
%
\newpage
\subsubsection*{Grafische Veranschaulichung}
\begin{figure}[h!]
\centering
\includegraphics[scale=0.5]{diagrams/Kurzfassung2}
\caption{Kurzfassung - grafische Veranschaulichung}
\end{figure}
%
%====================================================================================
%
\newpage
\subsection*{Projektergebnis}
{\yhbu
	Allgemeine Beschreibung, was vom Projektziel umgesetzt wurde, in einigen
	kurzen Sätzen. Optional Hinweise auf Erweiterungen. Gut machen sich in diesem
	Kapitel auch Bilder vom Gerät (HW) bzw. Screenshots (SW).
	Liste aller im Pflichtenheft aufgeführten Anforderungen, die nur teilweise oder gar
	nicht umgesetzt wurden (mit Begründungen).\\\\}
%	
\begin{itemize}
\item[A]Steuerung des Brushless-DC-Elektromotors
\item[B]Aufbau für Messung der Eingangsleistung
\item[C]Aufbau für Messung der Ausgangsleistung
\item[D]Aufbau für Messung der Drehgeschwindigkeit
\item[E]Messung des Drehmomentes
\item[F]
\item[G]
\item[H]Aufbau fer Steuerung für Schlittens der Wirbelstrombremse
\item[I]Webseite für Steuerung und Messwertentnahme
\item[J]Webseite für Bedienung 
\item[K]Speicherung der Messwerte in .csv-File
\item[L]
\end{itemize}
%
\paragraph*{Möglichkeiten zur Erweiterung}
\begin{itemize}
\item[Zu I]Darstellung der Messwerte in Diagrammform (HTML/CSS)
\end{itemize}
%
%====================================================================================
%
\newpage
\tableofcontents
%
%====================================================================================
%\clearpage\vfill\newpage{}
%====================================================================================
\cfoot{\thesection-\rightmark}	%setzt Kapitelnummer und Kapiteltext in die FusszeilenMitte
\noindent%
\newpage
\setcounter{section}{0}
\section{EINLEITUNG}
%	{\yhbu
%	In der Einleitung wird erklärt wieso man sich für dieses Thema entschieden hat.
%	(Zielsetzung und Aufgabenstellung des Gesamtprojekts, fachliches und
%	wirtschaftliches Umfeld)}
%
\paragraph*{Elektromotoren}\mbox{}\\
	Elektromotoren sind allgegenwärtig. Von der Entdeckung 1820 durch 'Hans Christian \O rsted' bis hin 
	zu den Teslas, die schon unsere heutigen Stra\ss en befahren. \\
	Sie sind aus dem heutigen Leben nicht mehr wegzudenken. 
	Darum ist es von höchstem Interesse ihr Funktionalität maximal ansnutzen zu können. 
	Ob nun hohe Drehzahl, gro\ss es Drehmoment, erhebliche Energieeffiezienz oder 
	geringer mechanischer 	Verschlei\ss \ das Ziel ist, all das geht allein von den Informationen, die über das beschriebene
	Gerät vorhanden sind, aus. 
	Von gro\ss er Wichtigkeit ist es darum Messungen mit hoher Genauigkeit und Fehlerfreiheit garantieren zu können. 
	Dies ist das Hauptziel der folgenden Diplomarbeit. 
%
\paragraph*{Zur Geschichte des Prüfstandes}\mbox{}\\
	Der vorliegende Elektromotorprüfstand entstand bereits im Schuljahr 2014/15 
	und wurde von einer Diplomarbeitsgruppe, bestehend aus Auer Christian, Fill Wolfgang Markus und Prost Claudio aus 
	der Abteilung Maschinenbau geplant und konstruiert. \\\\
%	
	Im Schuljahr 2016/17 wurde er dann von den Schülern Mahlknecht Lukas und Schöffmann Manuel 
	aus der Abschlussklasse 5AHWII unter Verwendung des Gerätes {\bf 'Texas Instruments - NI myDAC'} und Software {\bf 'LabVIEW'} weiterentwickelt. \\\\
%
	Jedoch waren zu diesem Zeitpunkt die Messergebnisse durch unkontrollierbare Spannungsspitzen in der Versorgung 
	des E-Motors stark fehlerbehaftet und deshalb unbrauchbar. \\
	Durch eine weiteroptimierung wird dieses Problem behoben. \\
	Au\ss erdem wird der gesamte Messvorgang automatisiert ablaufen um die Genauigkeit der Messungen zu steigern 
	und den durchführenden Prüfer zu entlasten.
%	
\paragraph*{Wahl dieser Diplomarbeit}\mbox{}\\
	Die weitere Optimierung des Elektromotorprüfstandes wurde ausgewählt, da sie die Verwendung von 
	Software sowie Hardware verbindet und ein weites Spektrum der in der bisherigen schulischen Laufbahn 
	gelernten Fähigkeiten zur Anwendung bringt. \\
	Zudem fanden beide Mitglieder der Dilpomarbeitsgruppe den theoretischen sowie praktischen Einsatz von 
	Elektromotoren äu\ss erst spannend. \\
%
%====================================================================================
%
\newpage
\subsection{Organisation der Diplomarbeit}
\begin{figure}[h!]
	\centering
	\includegraphics[scale=0.44, angle=90]{diagrams/Dipl_Organigramm5}
	\caption{Organisation der Diplomarbeit}
\end{figure}
%
%====================================================================================
%
\newpage
\subsection{Mindmap}
\begin{figure}[h!]
	\centering
	\includegraphics[scale=0.24, angle=90]{diagrams/Mindmap_Pruefstand3}
	\caption{Mindmap der Diplomarbeit}
\end{figure}
%
%====================================================================================
%
\newpage
\section{VERTIEFENDE AUFGABENSTELLUNG}
 \subsection{Simon Jehle}
Vertiefende Aufgabenstellung laut Antrag: {\bf Messdatenerfassung}
%
 \subsection{Patrick Krismer}
Vertiefende Aufgabenstellung laut Antrag: {\bf Ablaufsteuerung}
%
%====================================================================================
%
\newpage
\section{DOKUMENTATION DER ARBEIT}
   {\yhbu
	Es werden die Projektergebnisse dokumentiert.
	\begin{itemize*}
	\item	Grundkonzept
	\item	Theoretische Grundlagen
	\item	Praktische Umsetzung
	\item	Lösungsweg
	\item	Alternativer Lösungsweg
	\item	Ergebnisse inkl. Interpretation
	\end{itemize*}
	%
	Weitere Anregungen:
	\begin{itemize*}
	\item	Fertigungsunterlagen
	\item	Testfälle (Messergebnisse...)
	\item	Benutzerdokumentation
	\item	Verwendete Technologien und Entwicklungswerkzeuge
	\end{itemize*}
   }%yhbu
%
%====================================================================================
%====================================================================================
%====================================================================================
%
\newpage
\subsection{MS1 - Motoransteuerung}
%
\subsubsection{Drehzahleinstellung}
\begin{figure}[h!]
	\centering
	\includegraphics[width=\textwidth]{pics_simps/rpm_control}
	\caption{Blockdiagramm Drehzahleinstellung}
\end{figure}
%
\subsubsection{Programm ATMega8}
\begin{figure}[h!]
	\centering
	\includegraphics[width=\textwidth]{pics_simps/rpm_control_atmega8_code}
	\caption{Flussdiagramm zum Programmcode vom ATMega8}
\end{figure}
%
%====================================================================================
%
\newpage
\subsubsection{Eingangsleistungsmessung}
\paragraph{Messprinzip}\mbox{}\\
\begin{figure}[h!]
	\centering
	\includegraphics[width=\textwidth]{pics_simps/Eingangsleistungsmessung_aktivitaetsdiagramm}
	\caption{Eingangsleistungsmessung Prinzip}
\end{figure}
%
%====================================================================================
%
\newpage
\subsubsection{Schaltung}
\paragraph{Schaltungsentwurf}\mbox{}\\
\begin{figure}[h!]
	\centering
	\includegraphics[scale=0.65]{pics_simps/Schaltungsentwurf_Pin.pdf}
	\caption{Schaltungsentwurf für die Eingangsleistungsmessung}
\end{figure}
%
%====================================================================================
%
\newpage
\paragraph{Realisierung auf Lochrasterplatine}\mbox{}\\ %siehe https://tex.stackexchange.com/questions/32160/new-line-after-paragraph#32164
\begin{figure}[h!]
	\centering
	\includegraphics[width=\textwidth]{pics_simps/Schaltungsrealisierung_Pin}
	\caption{Foto der gelöteten Schaltung nach dem Schaltungsentwurf (+Vorverstärker für Ausgangsspannung) hier ohne Pumpkondensatoren für MAX680CPA+}
\end{figure}
%
\subsubsection{Arduino-Programm}
\begin{figure}[h!]
	\centering
	\includegraphics[width=\textwidth]{pics_simps/Programm_Pin}
	\caption{Flussdiagramm zum Programmablauf des Arduinos für die Eingangsleistungsmessung\vspace{0.3cm}\newline
	\textsuperscript{1}Daten von einer Waage zu empfangen ist wichtig für die Ausgangsleistungsmessung $\rightarrow$ Wird im nächsten Kapitel gezeigt}
\end{figure}
%
%====================================================================================
%====================================================================================
%
\newpage
\subsection{MS2 - Ausgangsleistungsmessung}
\subsubsection{Messprinzip}
\begin{figure}[h!]
	\centering
	\includegraphics[width=\textwidth]{pics_simps/Pout_Aktivitaetsdiagramm}
	\caption{Blockschaltbild des Prinzips der Ausgangsleistungsmessung}
\end{figure}
%
%====================================================================================
%
\newpage
\paragraph{Ausgangsleistung - Berechnung}\mbox{}\\
Die Ausgangsleistung kann mit der folgenden Formel der rotatorischen Leistung errechnet werden:\\
\begin{figure}[h!]
	\centering
	\huge$P\:=\:M\:*\:\omega$\vspace{0.5cm}\\
	\Large$M\:=l\:*\:F$\\
	\Large$\omega\:=\:2\,*\,\pi\,*\,f$\vspace{0.5cm}\\
	\raggedright
	\large P ... Leistung [W]\\
	\large M ... Drehmoment [Nm]\\
	\large $\omega$ ... Kreisfrequenz [$\frac{1}{s}$]\\
	\large l ... Länge des Kraftübertragungsarms von Drehmomentaufnehmer auf Waage [m]\\
	\large F ... Kraft (umgerechnet von Gewicht) [N]\\
	\large f ... Frequenz [$\frac{1}{s}$]
\end{figure}
%
%====================================================================================
%
\newpage
\subsubsection{Lichtschranke}
%
\paragraph{Aufbaudiagramm}\mbox{}\\
\begin{figure}[h!]
	\centering
	\includegraphics[width=\textwidth]{pics_simps/lichtschranke_prinzip}
	\caption{Aufbaudiagramm der Lichtschranke zur Drehzahlmessung}
\end{figure}
%
%====================================================================================
%
\newpage
\paragraph{Programmcode zur Drehzahlmessung}\mbox{}\\
\begin{figure}[h!]
	\centering
	\includegraphics[width=\textwidth]{pics_simps/lichtschranke_aktivitaetsdiagramm}
	\caption{Aktivitätsdiagramm Drehzahlmessung mit Lichtschranke}
\end{figure}
%
\paragraph{RS232- zu 5V bzw. Serial-Wandlerschaltung}\mbox{}\\
\paragraph{Schaltungsentwurf}\mbox{}\\
\begin{figure}[h!]
	\centering
	\includegraphics[width=\textwidth]{pics_simps/Schaltungsentwurf_232toSerial}
	\caption{Schaltungsentwurf für die RS232 zu Serial-Übersetzerschaltung}
\end{figure}
%
%====================================================================================
%
\newpage
\paragraph{Realisierung auf Steckbrett}\mbox{}\\
\begin{figure}[h!]
	\centering
	\includegraphics[width=\textwidth]{pics_simps/Schaltungsrealisierung_232toSerial}
	\caption{Foto der aufgebauten Schaltung nach Schaltungsentwurf}
\end{figure}
%
%====================================================================================
%
\newpage
\subsubsection{Kommunikationstest Waage - Arduino}
%
\paragraph{Testaufbau}\mbox{}\\
\begin{figure}[h!]
	\centering
	\includegraphics[width=\textwidth]{pics/waagetest364-circuit}
	\caption{Testaufbau der Kommunikation mit der Waage}
\end{figure}
%
\paragraph{Testergebnis}\mbox{}\\
\begin{figure}[h!]
	\centering
	\includegraphics[width=\textwidth]{pics/waagetest364}
	\caption{Vergleich Waage-Anzeige mit dem Serial-Monitor Output der Arduino-IDE}
\end{figure}
%
%====================================================================================
%
\newpage
\paragraph{Testprogramm}\mbox{}\\
\begin{figure}[h!]
	\centering
	\includegraphics[width=\textwidth]{pics/waagetest364-prog}
	\caption{Aktivitätsdiagramm für das Waage-Arduino Kommunikations-Testprogramm}
\end{figure}
%
%====================================================================================
%
\newpage
\subsection{MS2 - Wirbelstrom-Bremse}
Zuvor wurde der Bremssattel mit einen {\bf 'Schrack RL 306 024 - Koppelrelai'} gesteuert. 
Dieses wurde mit einem Tastknopf und Positionsendschalter geschalten. \\\\
Der grüne Tastknopf am Prüftisch startete den Bremsvorgang, 
der Positionsendschalter links lies die Polarität der Versorgung umschalten, 
sodass der Motor wieder rückwärts fuhr und der Positionsendschalterrechts beendete den Bremsvorgang. \\
%
\begin{figure}[h!]
  \centering
  \begin{minipage}[b]{0.4\textwidth}
    \includegraphics[width=\textwidth]{pics/bremse/pes_links_gross}
    \caption{Positionsendschalter links}
  \end{minipage}
  \hspace{10mm}
  \begin{minipage}[b]{0.4\textwidth}
    \includegraphics[width=\textwidth]{pics/bremse/pes_rechts}
    \caption{Positionsendschalter rechts}
  \end{minipage}
\end{figure}
%
\begin{figure}[h!]
Um genauere Messungen durchführen zu können musste der Bremssattel jedoch in bestimmten Abständen gestoppt werden können. 
Dies lies sich nicht mit dem Relai bewerkstelligen. \\
Stattdessen wird eine fertige H-Brücke-Platine 'MotoDriver2' verwendet, welche vom Arduino aus gesteuert wird. \\\\
Als Versorgungsspannung wird 12V verwendet, da hier eine Kennlinie im Datenblatt des DC-Motors vorliegt, und 
die Geschwindigkeit, in der der Schlitten fährt gut abzubremsen ist. So kann der Bremsweg bei den Mess-Stopps vernachlässig werden. \\\\
Die Positionsendschalter werden weiterhin verwendet und direkt mit dem Arduino verbunden. 
\end{figure}
%
%====================================================================================
%====================================================================================
%
\subsubsection*{Durchlaufzeit - Testmessung mit Relai}
\begin{figure}[h!]
\centering
\begin{tabular}{|c|c|c|c|c|}
\hline
\multicolumn{5}{|c|}{Zeitstop, ganzer Durchlauf}\\
\hline
\hline
Test Nr. & Vcc & Zeit bei 1 Magnet & Zeit bei 2 Magnet & Zeit bei 2.5 Magnet\\
\hline
Nr. 1 & 12V & 3.71 sec & 6.02 sec & 7.79 sec\\ 
Nr. 2 & 12V & 3.82 sec & 5.59 sec & 8.01 sec\\
Nr. 3 & 12V & 3.69 sec & 5.64 sec & 7.80 sec\\
\hline
\end{tabular}
\end{figure}
%
%====================================================================================
%
\newpage
\subsubsection*{Vorheriger Kabellaufplan mit Relai}
\begin{figure}[h!]
	\centering	
	\includegraphics[scale=0.35, angle=90]{pics/bremse/Bremse_Kabellaufplan_2}
	\caption{Vorheriger Kabellaufplan mit Relai}
\end{figure}
%
%====================================================================================
%
\newpage
\subsubsection*{Erneuerter Kabellaufplan mit H-Brücke}
\begin{figure}[h!]
	\centering	
	\includegraphics[width=\textwidth]{pics/bremse/Bremse_Kabellaufplan_erneuert}
	\caption{Erneuerter Kabellaufplan mit H-Brücke}
\end{figure}
%
%====================================================================================
%
\newpage
\subsubsection*{H-Brücke}
\begin{figure}[h!]
	\centering
	\includegraphics[scale=0.75]{pics/bremse/SBC-MotoDriver2}
	\caption{SBC-MotoDriver2}
\end{figure}
%
%====================================================================================
%
\newpage
\subsubsection*{H-Brücke-Gehäuse}
Um das MotoDriver2 zu schützen wird ein passendes Gehäuse mittels eines 3D-Druckers gefertigt. 
In diesem wird die Platine mit M3-Schrauben befestigt. \\
Mit M4-Schrauben wird der Deckel am Gehäuse festgeschraubt um es abzudecken. \\\\
%
An drei Seiten wurden Öffnungen freigelassen, um Kabel zur Versorgung (Front), Steuerung (Front) und 
Motorabgang (Links, Rechts), auch wenn der Deckel verschlossen ist, anschlie\ss en zu können. \\
%
\begin{figure}[h!]
	\centering
	\includegraphics[scale=0.4]{pics/bremse/gehaeuse1}
	\caption{Gehäuse1 - geschlossen}
\end{figure}
\begin{figure}[h!]
	\centering
	\includegraphics[scale=0.4]{pics/bremse/gehaeuse3}
	\caption{Gehäuse3 - geöffnet}
\end{figure}
%
%====================================================================================
%====================================================================================
%
\newpage
\subsection{MS3 - Serial Kommunikation zwischen Arduino und Raspberry Pi}
\subsubsection{Blockdiagramm}
\begin{figure}[h!]
	\centering
	\includegraphics[scale=1]{pics_simps/block_ardutoraspi}
	\caption{Blockdiagramm Verbindung Arduino zu Raspberry Pi mittels USB-Kabel}
\end{figure}
%
\subsubsection{Programme}
\paragraph{Arduino-Programm}\mbox{}\\
\begin{figure}[h!]
	\centering
	\includegraphics[scale=1]{pics_simps/ardutoraspi_aktivitaet}
	\caption{Aktivitätsdiagramm Arduino-Programm}
\end{figure}
%
%====================================================================================
%
\newpage
\paragraph{Raspberry-Programm}\mbox{}\\
\begin{figure}[h!]
	\centering
	\includegraphics[scale=1]{pics_simps/raspitoardu_aktivitaet}
	\caption{Aktivitätsdiagramm Raspi-Programm}
\end{figure}
%
%====================================================================================
%
\newpage
\subsubsection{Kommunikationstest}
\paragraph{Testaufbau}\mbox{}\\
\begin{figure}[h!]
	\centering
	\includegraphics[scale=0.42]{pics_simps/raspitoardutestaufbau_foto}
	\caption{Foto des Testaufbaus für die Kommunikation zwischen Arduino und Raspberry Pi}
\end{figure}
%
%====================================================================================
%
\newpage
\paragraph{Testergebnis}\mbox{}\\
\begin{figure}[h!]
	\centering
	\includegraphics[scale=0.7]{pics_simps/raspitoardutest_result}
	\caption{Ergebnis des Kommunikationstests von Arduino zu Raspi}
\end{figure}
%
%====================================================================================
%
\newpage
\subsection{MS3 - Datenbank bzw. Datenspeicherung}
\subsubsection*{CSV-Dateiformat}
Um die, vom Arduino gemessenen und verarbeiteten, Messwerte speichern zu können wird das 
Dateiformat {\bf CSV}(\underline{C}ommand \underline{S}eparated \underline{V}alues') verwendet. \\\\
%
Es beschreibt einen einfachen Aufbau einer Textdatei um strukturiert Daten speichern zu können. \\
Derzeit exestiert kein Standard, der dieses Dateiformat beschreibt, jedoch findet es weltweit Verwendung. 
So ist es möglich in und aus Tabellenkalkulations-Softwares wie {\it Microsoft Excel}, {\it LibreOffice Calc} 
und {\it weitere} ein .csv - File zu im- und exportieren um enthaltene Daten speichereffizient, selbst 
zwischen unterschiedlichen Betriebssystemen, übertragen zu können. \\\\
%
Ein Beispiel für eine .csv-Datei:
\begin{lstlisting}[frame=single]
Listen Nr., Alter, Name, Abteilunng
1, 17, Anton, El
2, 18, Berta, W
3, 18, Caesar, B
7, 16, Dora, ET
9, 17, Emil, MB
\end{lstlisting}
%
\vspace{10mm}
Der Aufbau funktioniert wie folgt:
%
\paragraph{Datensätze bzw. Zeilen}\mbox{}\\
Um Datensätze zu trennen wird ein Zeichen verwendet. Meistens wird dafür der {\bf Zeichenumbruch} benutzt. \\
Das Zeichen für den Zeichenumbruch kann unter verschiedenen Betriebssystemen variieren. 
%
\paragraph{Datenfeldern bzw. Spalten}\mbox{}\\
Um Datenfelder  zu trennen wird ein weiteres Zeichen verwendet. Meistens wird dafür das {\bf Komma} benutzt. \\
Weitere gebräuchliche Zeichen sind: {\it Semikolon ; }, {\it Doppelpunkt : }, {\it Tabulatorzeichen}, {\it Leerzeichen}. \\\\
%
Für den strukturellen Aufbau kann jedes belibige Zeichen verwendet werden, 
jedoch sollte es eindeutig sein und nicht in den Daten selbst vorkommen um Fehler zu vermeiden. 
%
\paragraph{Kopfzeile}\mbox{}\\
"Der erste Datensatz kann ein Kopfdatensatz sein, der die Spaltennamen definiert. "
%
%====================================================================================
%
\newpage
\subsection{MS4 - Messaktivierung}
\newpage
\subsection{MS4 - Website-Informationsinterface}
Der Messvorgang wird über eine Website gesteuet, die der Raspberry Pi als Webserver hostet. 
Von der Startseite aus kann der Messvorgang gesteuert werden, auf der Bedienungswebseite wird beschrieben, wie der 
Messvorgang vorbereitet und gesteuert wird. 
%
\subsubsection*{Startseite}
\begin{figure}[h!]
\centering
\includegraphics[width=\textwidth]{pics/website/startsite0}
\caption{Statseite 1/2}
\end{figure}
%
%====================================================================================
%
\newpage
\begin{figure}[h!]
\centering
\includegraphics[width=\textwidth]{pics/website/startsite_tabelle_leer}
\caption{Statseite 1/2}
\end{figure}
%
%====================================================================================
%
\newpage
\subsubsection*{Bedienwebseite}
\begin{figure}[h!]
\centering
\includegraphics[width=\textwidth]{pics/website/instrucionsite_1}
\caption{Bedienungswebseite 1/5}
\end{figure}
%
\begin{figure}[h!]
\centering
\includegraphics[width=\textwidth]{pics/website/instrucionsite_2}
\caption{Bedienungswebseite 2/5}
\end{figure}
%
%====================================================================================
%
\newpage
\begin{figure}[h!]
\centering
\includegraphics[width=\textwidth]{pics/website/instrucionsite_3}
\caption{Bedienungswebseite 3/5}
\end{figure}
%
\begin{figure}[h!]
\centering
\includegraphics[width=\textwidth]{pics/website/instrucionsite_4}
\caption{Bedienungswebseite 4/5}
\end{figure}
%
%====================================================================================
%
\newpage
\begin{figure}[h!]
\centering
\includegraphics[width=\textwidth]{pics/website/instrucionsite_5}
\caption{Bedienungswebseite 5/5}
\end{figure}
%
%====================================================================================
%
\newpage
\subsection{MS5 - Professionalisierung des Aufbaues mit Platinen und Bedienpult }
\subsection{MS5 - Automatischer Überstromschutz}
Der Überstromschutz ist ein Teilprogramm, welches kontrolliert, ob die gemessenen Werte in einem definierten Bereich sind und diesen nicht überschreiten. \\\\
{\bf Pseudocode Überstromschutz:}
\begin{lstlisting}[frame=single]
if Messwert > Ueberstrom then UeberstromZeit++
      else UeberstromZeit=0;
      if( UeberstromZeit > Zeitlimit) then NotAus();
\end{lstlisting}
%
%====================================================================================
%
\newpage
\subsection{Verwendete Technologien und Entwicklungswerkzeuge}
\subsubsection*{Software}
\begin{itemize}
\item Arduino IDE - Version 1.18.12
\item Code OSS - Version 1.42.1
\item GNU GCC - Version 9.2.1
\item Texmaker - Version 5.0.4
\item Firefox Browser - Version 74.0
\item GNU GIMP - Version 2.10.18
\item Libre Office Draw - Version 6.3.5.2
\item Draw.io - Version 12.8.8
\end{itemize}
%
\vspace{10mm}
\subsubsection*{Hardware}
%
\paragraph{Arduino MEGA2560}\mbox{}\\
\begin{figure}[h!]
	\centering
	\includegraphics[scale=0.25]{pics/arduinomega}
	\caption{Arduino MEGA Klon der Firma Elegoo mit ATMega2560 Mikrocontroller}
\end{figure}
%
%====================================================================================
%
\newpage
\paragraph{AD633JN Analogmultiplizierer}\mbox{}\\
\begin{figure}[h!]
	\centering
	\includegraphics[scale=0.5]{pics_simps/ad633jn}
	\caption{AD633JN Analogspannungsmultiplizierbaustein von Analog Devices}
\end{figure}
%
\paragraph{MAX680CPA+ Ladungspumpe}\mbox{}\\
\begin{figure}[h!]
	\centering
	\includegraphics[scale=1]{pics_simps/max680cpa}
	\caption{MAX680CPA+ Ladungspumpe zum erzeugen der +-10V die der AD633JN Baustein benötigt}
\end{figure}
%
%====================================================================================
%
\newpage
\paragraph{Kern 440-51N Präzisionswaage}\mbox{}\\
\begin{figure}[h!]
	\centering
	\includegraphics[scale=0.15]{pics_simps/440-51n}
	\caption{Kern 440-51N Präzisionswaage}
\end{figure}
%
\paragraph{Lichtschranke}\mbox{}\\
Es wurde eine zu diesem Zeitpunkt unbekannte, generische Lichtschranke genutzt.
%
\paragraph{Raspberry Pi}\mbox{}\\
\begin{figure}[h!]
	\centering
	\includegraphics[scale=0.7]{pics/raspi}
	\caption{Raspberry Pi 1 V1.2}
\end{figure}
%
%====================================================================================
%====================================================================================
%
\clearpage\vfill\newpage{}
%====================================================================================
\section{Erkl"arung der Eigenst"andigkeit der Arbeit}
	\noindent\\[0mm] EIDESSTATTLICHE ERKLÄRUNG
	\\[4mm]
	\parbox{152mm}{
	Ich erkläre an Eides statt, dass ich die vorliegende Arbeit selbständig und ohne
	fremde Hilfe verfasst, andere als die angegebenen Quellen und Hilfsmittel nicht
	benutzt und die den benutzten Quellen wörtlich und inhaltlich entnommenen
	Stellen als solche erkenntlich gemacht habe. Meine Arbeit darf öffentlich
	zugänglich gemacht werden, wenn kein Sperrvermerk vorliegt.
	}
	\\[19mm]\parbox{80mm}{
		\rule{60mm}{.5pt}\\
		\hspace*{3mm}Kappl, Datum
	}
	\parbox{80mm}{
		\rule{70mm}{.5pt}\\
		\hspace*{3mm}Simon Jehle
	}
		\\[19mm]\parbox{80mm}{
		\rule{60mm}{.5pt}\\
		\hspace*{3mm}Telfs, Datum
	}
	\parbox{80mm}{
		\rule{70mm}{.5pt}\\
		\hspace*{3mm}Patrick Krismer
	}
%	\\[19mm] \hspace*{3mm}\scalebox{1.7}{$\cdot\cdot\cdot$}

%
%====================================================================================
%====================================================================================
%
\newpage
\renewcommand{\thesection}{\Roman{section}\;}
\setcounter{section}{0}
%
\section{Abbildungsverzeichnis}
\vspace{-10mm}
\listoffigures
%	
%====================================================================================	
%	
\section{Tabellenverzeichnis}
\listoftables
%
%====================================================================================
%
\newpage
\section*{Beispiel Literaturverzeichnis}
%{\yhbu
	{\fontsize{10pt}{10pt}\selectfont
	(Übernommen aus dem Leitfaden des BMBF Reife- und Diplomprüfungen März 2014)
	\\[0mm]
	\begin{description*}
	\item[1. Werke eines Autors] Nachname, Vorname: Titel. Untertitel. -
		Verlagsort: Verlag, Jahr. Nachname,
		Vorname: Titel. Untertitel. Auflage - Verlagsort: Verlag, Jahr.
		\\[1mm]Beispiele:
		\\Sandgruber, Roman: Bittersüße Genüsse. Kulturgeschichte der Genußmittel. – Wien:
		Böhlau, 1986. Messmer, Hans-Peter: PC-Hardwarebuch. Aufbau, Funktionsweise,
		Programmierung. Ein Handbuch nicht nur für Profis. 2. Aufl. - Bonn: Addison-Wesley,
		1993.
		\vspace*{2mm}
	\item[2. Werke mehrerer Autoren] Nachname, Vorname; Nachname, Vorname; Nachname, Vorname: Titel.
		Untertitel. Auflage - Verlagsort: Verlag, Jahr.
		\\[1mm]Beispiel:
		\\Bauer, Leonhard; Matis, Herbert: Geburt der Neuzeit. Vom Feudalsystem zur
		Marktgesellschaft. - Mün- chen: Deutscher Taschenbuch Verlag, 1988.
		\vspace*{2mm}
	\item[3. Sammelwerke, Anthologien, CD-ROM mit Herausgeber] Nachname, Vorname (Herausgeber):
		Titel. Untertitel. Auflage - Verlagsort: Verlag, Jahr. Nachname, Vorname: Titel.
		Untertitel. In: Nachname, Vorname (Herausgeber): Titel. Untertitel. Auflage -
		Verlagsort: Verlag, Jahr.
		\\[1mm]Beispiele:
		\\Popp, Georg (Hg.): Die Großen der Welt. Von Echnaton bis Gutenberg. 3. Aufl. -
		Würzburg: Arena, 1979. Killik, John R.: Die industrielle Revolution in den Vereinigten
		Staaten. In: Adams, Willi Paul (Hg.): Die Vereinigten Staaten von Amerika. Fischer
		Weltgeschichte Bd. 30. - Frankfurt am Main: Fischer Taschenbuch Verlag, 1977. Killy,
		Walther (Hg.): Literatur Lexikon. Autoren u. Werke deutscher Sprache. – München:
		Bertelsmann, 1999. (Digitale Bibliothek, 2)
		\vspace*{2mm}
	\item[4. Mehrbändige Werke] Nachname, Vorname: Titel. Bd. 3 - Verlagsort: Verlag, Jahr.
		\\[1mm]Beispiel:
		\\Zenk, Andreas: Leitfaden für Novell NetWare. Grundlagen und Installation. Bd. 1 - Bonn:
		Addison Wesley, 1990.
		\vspace*{2mm}
	\item[5. Beiträge in Fachzeitschriften, Zeitungen] Nachname, Vorname des Autors des bearbeiteten
		Artikels: Titel des Artikels. In: Titel der Zeitschrift, Heftnummer, Jahrgang, Seite
		(eventuell: Verlagsort, Verlag).
		\\[1mm]Beispiel:
		\\Beck, Josef: Vorbild Gehirn. Neuronale Netze in der Anwendung. In: Chip, Nr. 7, 1993,
		Seite 26. - Würzburg: Vogel Verlag.
		\vspace*{2mm}
	\item[6. CD-ROM-Lexika]\hfill
		\\[1mm]Beispiel:
		\\Encarta 2000 - Microsoft 1999.
		\vspace*{2mm}
	\item[7. Internet] Nachname, Vorname des Autors: Titel. Online in Internet: URL: www-Adresse, Datum.
		(Autor und Titel wenn vorhanden, Online in Internet: URL: www-Adresse, Datum auf
		jeden Fall)
		\\[1mm]Beispiel:
		\\Ben Salah, Soia: Religiöser Fundamentalismus in Algerien. Online im Internet:
		URL: >>http:/\slash{}www.hausarbeiten.de\slash{}cgi-bin\slash{}superRD.pl<<,
		22.11.2000. Der Weg zur Doppelmonarchie.
		Online in Internet: URL:
		http:/\slash{}www.parlinkom.gv.at\slash{}pd\slash{}doep\slash{}d-k1-2.htm,
		22.11.2000.
		\vspace*{2mm}
	\item[8. Firmenbroschüren, CD-ROM] Werden Inhalte von Firmenunterlagen verwendet,
		dann ist ebenfalls die Quelle anzugeben.
		\\[1mm]Beispiel:
		\\Digitale Turbinenregler. Broschüre der Firma VOITH-HYDRO GmbH, 2012.
		\vspace*{2mm}
	\item[9. Abbildungen, Pläne] Werden Abbildungen aus einer fremden Quelle
		[z.B. Download, Scannen) in die Diplomarbeit eingefügt,
		so ist unmittelbar darunter die Quelle anzugeben.
		\\[1mm]Beispiel:
		\\Abb. 1: Digitaler Turbinenregler [ANDRITZ HYDRO]
		\vspace*{2mm}
	\item[10. Persönliche Mitteilungen]\hfill
		\\[1mm]Beispiel:
		\\Persönliche Mitteilung durch: König, Manfred:
		Kössler GmbH Turbinenbau am 8. März 2013.
	\end{description*}
	}
%	}%yhbu
	%
	%
	%
	%
	%
	%
	%
	%
%
%====================================================================================
%
\newpage
\section{Literaturverzeichnis}
	\begin{description*}
{\footnotesize
\subsection*{Datenblätter:}
%
	\item[\bf Bremsmotor]:\\
		https://www.conrad.com/p/drive-system-europe-dc-gearmotor-dsmp320-12-0014-bf-192498-12-v-dc-053-a-008-nm-373-rpm-shaft-diameter-6-mm-1-pcs-192498\\
		zuletzt abgerufen: 20.03.2020\\\\
%
	\item[\bf MotoDriver2]:\\
		https://www.conrad.at/de/p/entwickler-platine-sbc-motodriver2-arduino-banana-pi-cubieboard-pcduino-raspberry-pi-raspberry-pi-2-b-raspberry-1573541.html\\
		zuletzt abgerufen: 20.03.2020\\\\
%
\subsubsection*{Internet:}
%
	\item[CSV-Dateiformat]:\\
	Wikipedia, Diverse Autoren: "CSV (Dateiformat)"\\
	\url{https://de.wikipedia.org/wiki/CSV\_(Dateiformat)}\\
	(zuletzt aufgerufen: 26 März 2020)\\\\
%
	Yakov Shafranovich <ietf@shaftek.org>: Common Format and MIME Type for Comma-Separated Values (CSV) Files\\
	\url{https://tools.ietf.org/html/rfc4180}\\
	(zuletzt aufgerufen: 28 März 2020)\\\\
%
	\item[hallo]:\\
%
\subsection*{Persönliche Mitteilungen:}
		Persönliche Mitteilung durch: HTL-Lehrer XX XX: Begriff des Bremssattels am XX Februar 2020. \\\\
%
	\item[3.]:
	\item[4.]:
	
\subsection*{Vorherige Diplomarbeit}
\item[•]Mahlknecht Lukas und Schöffmann Manuel - 5AHWII (2016/17)\\
}
\end{description*}
%
%====================================================================================	
%
\newpage
\section{Abk"urzungs- und Symbolverzeichnis}
%	\vspace*{-6mm}\rotatebox{2}{\parbox{120mm}{\korr
%	iXH h"atte doch gleich ein {\em\dq Glossar\dq} (Begriffserkl"arung) erstellt (nein???)[XH]}}	
\begin{figure}[h!]
Dipl. Ing. \dotfill Diplomingeneur \\
etc. \dotfill etcetera\\
zb. \dotfill zum Beispiel\\
usw. \dotfill und so weiter\\
%u.w. \dotfill und weitere\\
%dt. \dotfill zu Deutsch\\
Raspi \dotfill Raspberry Pi B+ V1.2\\
Arduino \dotfill Arduino Mega 2560\\
Vcc \dotfill Versorgungsspannung\\
\dotfill\\
\end{figure}
%
%====================================================================================
%====================================================================================
%
%\clearpage\vfill\newpage{}
%====================================================================================
%
\renewcommand{\thesection}{\arabic{section}}
\newpage
\section{Anhang}
%	\noindent\\[-2mm]
%	\hspace*{3mm}{\sc\textbf{\Large Anhang}}
%	\\[1mm]\hspace*{20mm}\parbox{100mm}{\korr
%		iXH w"urde hier einen neuen Abschnitt (\tbs{}part\{Anhang\}) beginnen}
	%\noindent\\[-5mm]
	%
	%
\appendix
\renewcommand{\thesection}{\Alph{section}}
\setcounter{section}{1}
\setcounter{subsection}{0}
 \subsection{Pflichtenheft (optional)}
	{\yhbu Zur Umsetzung des Projektzieles werden messbare Kriterien formuliert.}
 \subsection{Schlussfolgerung / Projekterfahrung}
	Krismer: Bin zu bled einfachste Sachen auf die Reihe zu bringen. \\
	Krismer (über Jehle): Er kriegts hin. Isch a guater Mann. \\
 \subsection{Projektterminplanung}
	\noindent{\yhbu
	Screenshots der MS Project-Datei. Die Ausgabe muss lesbar sein (eventuell auf
	mehrere Bilder verteilen). Insbesondere ist darauf zu achten, dass die Zeitachse
	und die Vorgangsachse auf jedem Bild sichtbar sind! Es muss nicht MS-Project
	verwendet werden!
	\\[2mm]
	Projektbalkenplan (Gantt-Diagramm)
	\\[2mm]
	Excel
	}
	
\paragraph{Meilensteine} \mbox{}\\
\begin{figure}[h!]
	\centering	
	\footnotesize
	\begin{tabular}{|c|c|c|c|}
		\hline
		Datum & Meilenstein & Kandidat & Beschreibung\\
		\hline \hline 
		05.11.2019 & MS1 & Simon Jehle & Pflichtenheft, Grobdesign, \\&&& Testplan, Eingangsleistungsmessung (Motor) \\
		\hline
		05.11.2019 & MS1 & Patrick Krismer & Pflichtenheft, Grobdesign, Testplan, \\&&&Motoransteuerung (Drehzahlregulierung) \\
		\hline	
		17.12.2019 & MS2 & Simon Jehle & Leistungsabgabemessung (Waage) \\
		\hline
		17.12.2019 & MS2 & Patrick Krismer & Wirbelstrombremssteuerung \\
		\hline
		14.01.2020 & MS3 & Simon Jehle & Kommunikation zwischen \\&&& Raspberry Pi und Arduino \\
		\hline
		14.01.2020 & MS3 & Patrick Krismer & Datenbank (Raspberry Pi) \\
		\hline
		18.02.2020 & MS4 & Simon Jehle & Website-Informationsinterface (Raspberry Pi), \\&&& DS fertig \\
		\hline
		18.02.2020 & MS4 & Patrick Krismer & Messaktivierung (Raspberry Pi), \\&&& DS fertig \\
		\hline
		17.03.2020 & MS5 & Simon Jehle & Professionalisierung des Aufbaues \\&&& mit Platinen und Bedienpult \\
		\hline
		17.03.2020 & MS5 & Patrick Krismer & Automatisierte Überstromabschaltung \\&&& des Motors \\
		\hline
%		\caption{Terminplanung - Meilensteine}
	\end{tabular}
\end{figure}
%	
%====================================================================================
%
\newpage
 \subsection{Arbeitsnachweis Diplomarbeit}
	\noindent{\yhbu
	Dieser erfolgt durch ständige Aufzeichnungen der Schüler im Projekttagebuch.
	\\[1mm]
	Für jeden Projektmitarbeiter wird eine Tabelle gemäß Muster ausgefüllt. In dieser
	Aufzeichnung werden auch die Unterrichtsprojektanteile, die in die Arbeit
	eingeflossen sind \;
	\scalebox{3}{\begin{picture}(0,0)\put(-1.6,0){\color{purple}\circle{4}}\end{picture}}
	aufgezeigt.
	}
	\paragraph{\color{teal}\scriptsize Tabelle: Arbeitsaufstellung}\noindent\\[-3mm]
	{\fontsize{9pt}{9pt}\selectfont
	\\
	\noindent
	\begin{tabular}{|l|l|l|p{80mm}|l|}
	\hline
	\multicolumn{5}{|c|}{\parbox{4em}{\hfill\\[-0mm]\color{dkbu}Jehle}}	\\
	\hline
	Datum	&Uhrzeit	&\parbox{4em}{\hfill\\[-0mm]Stunden\\nn:nn\vspace*{1mm}}
					&Beschreibung	&Betreuer	\\
	\hline
	01.11.2018	&08:00–11:30
				& &Was wurde gemacht (eine Zeile!) &	\\
	\hline
	& & & &\\
	\hline
	& & & &\\
	\hline
	& & SUMME & &	\\
	\hline
	\end{tabular}
%	
%	
	\vspace{10mm}
	\begin{tabular}{|l|l|l|p{80mm}|l|}
	\hline
	\multicolumn{5}{|c|}{\parbox{4em}{\hfill\\[-0mm]\color{dkbu}Krismer}}	\\
	\hline
	Datum	&Uhrzeit	&\parbox{4em}{\hfill\\[-0mm]Stunden\\nn:nn\vspace*{1mm}}
					&Beschreibung	&Betreuer	\\
	\hline
	01.11.2018	&08:00–11:30
				& &Was wurde gemacht (eine Zeile!) &	\\
	\hline
	& & & &\\
	\hline
	& & & &\\
	\hline
	& & SUMME & &	\\
	\hline
	\end{tabular}
	}

 \subsection{Datenbl"atter (optional)}
	\noindent{\yhbu
	Meist sind die Datenblätter sehr umfangreich, daher werden im Anhang nur die
	notwendigen Bereiche dargestellt, auf der CD wird das vollständige Datenblatt
	gespeichert
	}

 \subsection{Technische Zeichnungen (optional)}
	\noindent\\[10mm]
	%
	\noindent{\yhbu
	{\bf Abzugeben sind:}
	\xilist{}{4}{-2}{1}{0}{
	\item[1] gebundene Dokumentationen mit Deckblatt (Format: A4) – für die Bibliothek
		(die HTL Bindung ist zu verwenden!!)
	\item[1] Korrekturversion (Form und Aussehen mit Projektbetreuer vereinbaren)
	\item[1] Version für die Firma (optional -- mit Betreuer vereinbaren)
	\item[2] CDs mit allen Unterlagen (Word, Bilder, Code...)
	}
	}
%	
%====================================================================================
%====================================================================================
%
\label{LastPage}
%\addtocontents{toc}{\protect\end{multicols}}
\end{document}